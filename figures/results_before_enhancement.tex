
\begin{figure*}[!t]
  \centering

 \subfigure[Dale Narrow Trend-Preserving ]{
  %left, bottom, right and top: clip,trim=0.1in 0.1in 0.1in 0.1in,
    \includegraphics[clip,trim=0.8in 0.8in 1.8in 1.0in,width=0.235\textwidth]{"images/before enhance_S_rel/New Narrow TP Datasets"}
  %left, bottom, right and top: 
    \includegraphics[clip,trim=0.8in 0.8in 1.8in 1.0in,width=0.235\textwidth]{"images/before enhance_S_rec/New Narrow TP Datasets"}
    \label{fig:enhancement sub(a)} 
}
\subfigure[UniBic Narrow ]{
  %left, bottom, right and top: clip,trim=0.1in 0.1in 0.1in 0.1in,
    \includegraphics[clip,trim=0.8in 0.8in 1.8in 1.0in,width=0.235\textwidth]{"images/before enhance_S_rel/UniBic Narrow Datasets"}
  %left, bottom, right and top: clip,trim=0.1in 0.1in 0.1in 0.1in,
    \includegraphics[clip,trim=0.8in 0.8in 1.8in 1.0in,width=0.235\textwidth]{"images/before enhance_S_rec/UniBic Narrow Datasets"}
    \label{fig:enhancement sub(b)}
}\\
\subfigure[Dale Square Trend-Preserving]{
  %left, bottom, right and top: clip,trim=0.1in 0.1in 0.1in 0.1in,
    \includegraphics[clip,trim=0.8in 0.8in 1.8in 1.0in,width=0.235\textwidth]{"images/before enhance_S_rel/New Square TP Datasets"}
  %left, bottom, right and top: clip,trim=0.1in 0.1in 0.1in 0.1in,
    \includegraphics[clip,trim=0.8in 0.8in 1.8in 1.0in,width=0.235\textwidth]{"images/before enhance_S_rec/New Square TP Datasets"}

}
 \subfigure[UniBic Square Trend-Preserving (Type I)]{
  %left, bottom, right and top: clip,trim=0.1in 0.1in 0.1in 0.1in,
    \includegraphics[clip,trim=0.8in 0.8in 1.8in 1.0in,width=0.235\textwidth]{"images/before enhance_S_rel/UniBic Square TP"}
  %left, bottom, right and top: clip,trim=0.1in 0.1in 0.1in 0.1in,
    \includegraphics[clip,trim=0.8in 0.8in 1.8in 1.0in,width=0.235\textwidth]{"images/before enhance_S_rec/UniBic Square TP"}

    }
     \caption[]{Results of comparative evaluation using new benchmark generalized Trend-Preserving (TP) datasets (Dale Narrow TP \& Dale Square TP) vs. the UniBic benchmark data (Narrow \& Square TP). The MOEA method utilized ASR as the fitness measure.}
       \label{fig:TP and unibic compara}
    \end{figure*}
    

\begin{figure}[!t]
    \subfigure[Fabia Noisy Datasets]{
  %left, bottom, right and top: clip,trim=0.1in 0.1in 0.1in 0.1in,
    \includegraphics[clip,trim=0.8in 0.8in 1.8in 1.0in,width=0.23\textwidth]{"images/Noisy_"}    }
  %left, bottom, right and top: clip,trim=0.1in 0.1in 0.1in 0.1in,
  \subfigure[Fabia Noisy Free Datasets]{
    \includegraphics[clip,trim=0.8in 0.8in 1.8in 1.0in,width=0.23\textwidth]{"images/Noise Free_"}}
    
    \caption[]{Comparative evaluation of Biclustering Methods on FABIA (Noisy and noise free) datasets. MOEA utilized ASR as its fitness measure. }
\label{fig:Before enhancement FIBIA}
\end{figure}

\begin{figure}[!t]
  \centering
 \subfigure[Dale Narrow TP]{
  %left, bottom, right and top: clip,
  \includegraphics[width=3.8cm, height=3.9cm]{"images/before enhance_NSGA/New Narrow TP Datasets"}
}
\subfigure[UniBic Narrow ]{
  %left, bottom, right and top: clip,trim=0.1in 0.1in 0.1in 0.1in,
    \includegraphics[width=3.8cm, height=3.9cm]{"images/before enhance_NSGA/UniBic Narrow Datasets"}
}
\subfigure[Dale Square TP]{
  %left, bottom, right and top: clip,trim=0.1in 0.1in 0.1in 0.1in,
    \includegraphics[width=3.8cm, height=3.9cm]{"images/before enhance_NSGA/New Square TP Datasets"}
}
 \subfigure[UniBic Square TP (Type I) ]{
  %left, bottom, right and top: clip,trim=0.1in 0.1in 0.1in 0.1in,
    \includegraphics[width=3.8cm, height=3.9cm]{"images/before enhance_NSGA/UniBic Square TP"}
    }
\caption[]{Comparative performance of MOEA using 3 fitness functions ASR, SCS and VE$^T$ on generalized Trend-Preserving (TP) datasets (Dale Narrow TP \& Dale Square TP) vs. the UniBic benchmark data.}
 \label{fig: ALL datasets NSGA_aa}.
    \end{figure}
    
\begin{figure}[!t]
\subfigure[Column Constant (Type II) ]{
  %left, bottom, right and top: clip,trim=0.1in 0.1in 0.1in 0.1in,
    \includegraphics[width=3.8cm, height=3.9cm]{"images/before enhance_NSGA/UniBic Square Column Const"}
}
\subfigure[Row Constant (Type III)]{
  %left, bottom, right and top: clip,trim=0.1in 0.1in 0.1in 0.1in,
    \includegraphics[width=3.8cm, height=3.9cm]{"images/before enhance_NSGA/UniBic Square Row Const"}
} 
\subfigure[Shift-Scale (Type IV)]{
%left, bottom, right and top: clip,trim=0.1in 0.1in 0.1in 0.1in,
\includegraphics[width=3.8cm, height=3.9cm]{"images/before enhance_NSGA/UniBic Square Shift-Scale"}
}
\subfigure[Shift (Type V)]{
  %left, bottom, right and top: clip,trim=0.1in 0.1in 0.1in 0.1in,
    \includegraphics[width=3.8cm, height=3.9cm]{"images/before enhance_NSGA/UniBic Square Shift"}
    }
\subfigure[Scale (Type VI) ]{
  %left, bottom, right and top: clip,trim=0.1in 0.1in 0.1in 0.1in,
    \includegraphics[width=3.8cm, height=3.9cm]{"images/before enhance_NSGA/UniBic Square Scale"}
    }
\subfigure[ Overlap]{
\includegraphics[width=3.8cm, height=3.9cm]{"images/before enhance_NSGA/UniBic Overlap"}}
\subfigure[FABIA Noisy]{
  %left, bottom, right and top: clip,trim=0.1in 0.1in 0.1in 0.1in,
    \includegraphics[width=3.8cm, height=3.9cm]{"images/before enhance_NSGA/Noisy"}
    }
    \subfigure[FABIA Noise Free]{
  %left, bottom, right and top: clip,trim=0.1in 0.1in 0.1in 0.1in,
    \includegraphics[width=3.8cm, height=3.9cm]{"images/before enhance_NSGA/Noise Free"}
    }
\caption[]{Comparative performance of 3 fitness functions ASR, SCS and VE$^T$ with MOEA on  UniBic and FABIA  benchmark datasets. }
 \label{fig: ALL datasets NSGA}
\end{figure}

