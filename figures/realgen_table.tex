
% \begin{table*}[!t]
% \centering
% \caption{The improvement percentage of Symmetric Relevance by using ASR, SCS, TPC and VEt on four algorithm. }
% \label{tab: enhancement framework}
% \begin{tabular}{lllll||lllll}
% \toprule
%     & \multicolumn{4}{l}{Symmetric Relevance}       & \multicolumn{4}{l}{Symmetric Recovery}           \\\midrule
% & BicPAMS & CC  & ISA    & UniBic & BicPAMS & CC & ISA     & UniBic          \\


% ASR     & 6.7\%        & 2.3\%  & 7.0\%  & 10.4\% & -30.4\%     & -19.9\% & -27.8\% & -26.3\% \\
% SCS     & 5.1\%        & 3.0\%  & 3.2\%  & 20.1\% & -30.6\%     & -18.8\% & -26.3\% & -19.0\% \\
% TPC     & 6.3\%        & 2.2\%  & 3.5\%  & 20.3\% &-31.5\%     & -19.4\% & -28.1\% & -18.1\% \\
% VEt     & 3.6\%        & -2.5\% & 1.6\%  & 19.5\% &-31.7\%     & -23.6\% & -29.6\% & -24.7\% \\\bottomrule
% \end{tabular}
% \end{table*}


\begin{table}[!t]
\centering
   \caption{Description of Real Gene Datasets and the Mean No. of Biclusters returned by each Algorithm}
    \label{tab:description of real gene }
    \scalebox{0.55}{
\begin{tabular}{lll||lllllllll}
 \toprule
                      &            &              & \multicolumn{9}{l}{Mean No. of biclusters returned by each algorithm}                     \\
                             Datasets & \makecell{No.of \\Genes} &  \makecell{No.of \\Samples} &\ BicPAMS & CC & EvoBexpa & FABIA & ISA & MOEA & OPSM & PPM & UniBic \\   \midrule
breast                & 1213       & 97           & 182     & 20          & 5        & 10    & 562 & 5       & 12   & 10             & 100    \\
dlbcl      & 661        & 180          & 189     & 20          & 5        & 10    & 313 & 5       & 19   & 10             & 100    \\
multi       & 5565       & 102          & 229     & 20          & 5        & 10    & 461 & 5       & 12   & 10             & 100   \\\bottomrule
\end{tabular}}
\begin{tablenotes}
		\tiny
		\item ${breast}$: breast cancer’ dataset was aimed at a predictive gene signature for the outcome of a breast cancer therapy.\\
		${dlbcl}$ diffuse large-B-cell lymphoma dataset was aimed at predicting the survival after chemotherapy.\\
	    ${multi}$: multiple tissue types dataset are gene expression profiles from human cancer samples from diverse tissues
and cell lines.
	\end{tablenotes}
\end{table}
