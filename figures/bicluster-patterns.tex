
\begin{figure}
    \centering
    \subfigure[\footnotesize {No Pattern}]{\includegraphics[width=0.56in]{"images/bicluster_models/no pattern"}}\qquad
  %  \subfigure[Constant Bicluster]{\includegraphics[width=0.6in]{"images/bicluster_models/constant"}}\qquad
    \subfigure[Constant Rows] {\includegraphics[width=0.56in]{"images/bicluster_models/constant rows"}}\qquad
    \subfigure[Constant Columns]{\includegraphics[width=0.56in]{"images/bicluster_models/constant cols"}} \qquad
    \subfigure[Additive Pattern]{\includegraphics[width=0.56in]{"images/bicluster_models/additive"}}\qquad\\
    \subfigure[Multiplicative Pattern]{\includegraphics[width=0.56in]{"images/bicluster_models/multiplicative"}}\qquad
    \subfigure[Additive Multiplicative ]{\includegraphics[width=0.56in]{"images/bicluster_models/additive multiplicative"}}\qquad
    \subfigure[Order Preserving]{\includegraphics[width=0.56in]{"images/bicluster_models/order preserving"}}\qquad
    \subfigure[Trend Preserving]{\includegraphics[width=0.56in]{"images/bicluster_models/trend preserving"}}\qquad
    \caption{
        Visualization of different bicluster models with size $30 \times 30$.
        The vectors $a$, $b$, $p$, and $q$ as well as the constant $\pi$ were generated uniformly at random and reused for each pattern shown.
        To show the different models, the appropriate vector(s) were set to zeros or ones as required, e.g. constant rows required $b$ and $q$ set to ones and $p$ set to zeros.
        The ``No Pattern'' bicluster is simply a $30 \times 30$ matrix of uniformly random numbers to emphasize the inherent plaid pattern of the bicluster models.
    }
    \label{fig:biclustermodels}
\end{figure}
