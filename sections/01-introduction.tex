%  \IEEEraisesectionheading{
% \section{Introduction}
% \label{sec:introduction}}
\section{Introduction}
\label{sec:introduction}
A key challenge in biological data analysis is  characterization of genetic factors that underlie phenotypic differences. 
Microarrays can be used
to collect a large amount of gene expression information,
which can be used to profile various kinds of biological
changes~\cite{huang2012parallelized}.
Such data hold great potential to understand how particular genes are expressed in an organism's phenotype, and also how these genes contribute to the organism's susceptibility to a disease or trait.
An understanding of how to organize, process, and transform these data into useful domain knowledge is crucial.
Unsupervised learning models are useful in understanding patterns in data.
In gene expression data analysis, the overall hypothesis is that small sets of genes may drive multiple cellular processes and not necessarily span the entire condition subspace. %\cite{xu2011bartmap}.
Hence, one-dimensional clustering methods may not uncover optimal/useful local condition patterns of genes.
Biclustering offers a solution by performing simultaneous clustering in two dimensions. It integrates feature selection and clustering without any prior information.

The relationship between clusters of genes (rows, features) and clusters of samples (columns, conditions) are established in the biclustering process. A bicluster is a submatrix with a subset of rows behaving similarly across a subset of columns.
%The goal of a biclustering algorithm is to identify a set of biclusters with pairs of row and column subsets.
Given a gene expression matrix $M$, the entries of a single bicluster $B$ referenced by the set of genes $I$ and set of samples $J$ is given by
\begin{equation}
	B_{ij} = \left\{ M_{ij}\ \mathrm{such\ that}\ i\in I\ \mathrm{and}\ j\in J \right\}
\end{equation}
 such that the entries indexed by the Cartesian product of these sets exhibit a distinct correlated pattern.
Biclustering algorithms perform local clustering on subset of genes and conditions with the objective of identifying an optimal set of biclusters $B$. These biclusters are useful for discovering patterns of co-regulated/co-expressed genes across a subset of samples in gene expression data~\cite{pontes2015biclustering}.
Biclustering is an NP hard problem. Thus the only perfect biclustering algorithm is an exhaustive search, which is computationally intractable on large datasets.
In the past two decades, there has been an influx of proposed biclustering algorithms, as reviewed in~\cite{prelic2006systematic, eren2012comparative, oghabian2014biclustering, pontes2015biclustering, roy2016analysis}.
Multi-objective optimization~\cite{deb2014multi} techniques are suited for addressing the challenge of identifying the global optima in very large search spaces with potentially conflicting objectives. 

This work applies a multi-objective evolutionary algorithm (MOEA) framework to address the biclustering problem. The MOEA framework is based on the Non-dominated Sorting Genetic Algorithm (NSGA-II)~\cite{deb2002fast}, combined with a novel local search hill climber tailored for the biclustering problem. The NSGA-II algorithm, being an MOEA, can support any number of fitness functions and return a set of Pareto-optimal biclusters.
This algorithm differs from other evolutionary based biclustering algorithms~\cite{pontes2013configurable, mitra2006multi} both in fitness measure and in post-processing methods.
In addition, this paper addresses generation of appropriate benchmark data for unsupervised learning applications and improvement of existing external validation metrics.
The novel synthetic data generation tool is useful for generating biclusters of varying distributions for a wide range of trend-preserving standard benchmark data. This is useful for evaluate performance of biclustering algorithms.
%In unsupervised learning applications such as gene expression data analysis, where ground truth is not readily available, it is valuable to have useful tools for simulation of synthetic data.
%The main contributions of this work are as follows:
%\begin{enumerate}
    %%% Added by Jeff - 2019.04.26 %%%
    %\item Propose a framework for an effective evolutionary based biclustering algorithm using NSGA-II and bicluster fitness functions.
 %   \item Provide a ``drag and drop'' toolset for turning internal bicluster quality measures into an effective evolutionary based biclustering algorithm using NSGA-II.
    %%%%%%%%%%%%%%%%%%%%%%%%%%%%%%%%%%
   % \item Present a methodology for generating a gold standard benchmark biclustering datasets. These simulate trend-preserving datasets that can be replicated and transformed to different structures to evaluate performance of biclustering algorithms.
  %  \item Improve the robustness of existing validation metrics. 
%
%\end{enumerate}

%%% Added by Jeff - 2019.04.26 %%%
%The results obtained demonstrate the effectiveness of the MOEA approach for discovering useful biclusters on a varied set of synthetic datasets when applied with the average Spearman's rho (ASR) measure as the fitness function.
The results obtained demonstrate the potential of internal bicluster quality measures to guide NSGA-II to high quality biclusters, especially when applied with Average Spearman's Rho as the fitness function. The python implementation of the method is available on our GitHub repository: https://github.com/clslabMSU.
%%%%%%%%%%%%%%%%%%%%%%%%%%%%%%%%%%