\section{Related Work}
\label{sec:related}
%Discuss different biclustering algorithms proposed... strengths and weaknesses. How does this work differ?

%As mentioned in Section~\ref{sec:introduction}, there are at least 30 biclustering methods present in the literature, with more being devised all the time. 
%There have been several biclustering algorithms proposed over the past two decades as reviewed in  \cite{prelic2006systematic,eren2012comparative,oghabian2014biclustering,pontes2015biclustering,roy2016analysis}.
Though multiple EBMs have been proposed in literature, only very few are readily available for implementation such as Evo-Bexpa. 
The Evolutionary Biclustering based in Expression Patterns (Evo-Bexpa) method ~\cite{pontes2013configurable} combines four criterion (transposed virtual error (VE$^T$), bicluster volume, overlap, gene variance) into a single aggregate objective function by a weighted summation. The configuration of the weights is based on the domain application. For superior performance, it needs to incorporate user preferences based on some prior knowledge regarding the results.

For the empirical evaluation, other non-EBMs with readily available implementation are utilized for comparason. They are briefly described in Table~\ref{tab:algsummary}. 
These include Cheng \& Church (CC) \cite{cheng2000biclustering}, Iterative Signature Algorithm (ISA) \cite{bergmann2003iterative}, Order-Preserving Submatrices Algorithm (OPSM) \cite{ben2003discovering}, Factor Analysis for Bicluster Acquisition (FABIA)~\cite{hochreiter2010fabia}, Penalized Plaid Model (PPM)~\cite{chekouo2015thepenalized}, UniBic~\cite{wang2016unibic}, and Biclustering based on PAttern Mining Software (BicPAMS)~\cite{henriques2017bicpams}.
%These algorithms, with readily available implemention, span over two decades and are summarized in Table~\ref{tab:algsummary}. 
%We focused on these methods, as their implementation were readily available. 
Given that UniBic is an extension/improvement of the graph-based biclustering method QUBIC~\cite{li2009qubic}, it isn't include QUBIC in the experiments, though it's implementation was publicly available. Likewise, only BicPAMS~\cite{henriques2017bicpams} is included given that it is the most recent improved version of prior pattern mining biclustering algorithms since the initial one: BicPAM \cite{henriques2014bicpam}.
%, BicSPAM \cite{henriques2014bicspam}, BiP \cite{henriques2015biclustering}, and BicNET \cite{henriques2016bicnet}).

 \begin{table} [!t]
	\caption{Summary of Biclustering Algorithms Utilized for Comparison}
	\label{tab:algsummary}
	\centering
	\scalebox{0.8}{
	\begin{tabular}{p{2cm}p{6cm}p{2cm}}
		\toprule
		Algorithm (Year) & Description & Implementation Source \\
		\midrule
        CC (2000)
        & Deterministic greedy algorithm that finds biclusters by minimizing the Mean Squared Residue (MSR) score of a discovered submatrix
        & Python \cite{Eren2013} \\ \\

        ISA (2003)
        & Non-deterministic method that discovers biclusters in an iterative manner, even in the presence of noise and overlapping biclusters.
        & R package \texttt{isa2} \cite{csardi2010modular} \\ \\
        
		OPSM (2003)
		& Deterministic method based on an optimal reordering approach of using a probabilistic model to describe the biclusters.
		& BicAT \cite{barkow2006bicat} \\ \\
		
		FABIA (2010)
		& A generative multiplicative model for discovering biclusters by assuming a non-Gaussian signal distributions with heavy tails.
		& Bioconductor \texttt{fabia} \cite{hochreiter2010fabia} \\ \\
		
		Evo-Bexpa (2013)
		& An evolutionary computing approach to the biclustering problem.
		  Fitness function is based on transposed virtual error, degree of overlap, and gene variance.
		& Python$^1$ \cite{pontes2013configurable} \\ \\
		
		PPM (2015)
		& A modified extended version of the Bayesian biclustering model.
		  The PPM method fully accounts for a general overlapping structure.
		  Parameters in the Penalized Plaid model are found all at once by a dedicated Markov chain Monte Carlo sampler.
		& Java \cite{chekouo2015thepenalized} \\ \\
		
		UniBic (2016)
		& Applies longest common subsequence algorithm to translate the input data matrix to a rank matrix such that the rows are discretized as rank vectors.
		& C$^2$ \cite{wang2016unibic} \\ \\
		
		BicPAMS (2017)
		& Aggregate of state-of-the-art pattern mining approaches in biclustering.
		& Java$^3$ \cite{henriques2017bicpams} \\
		
		\bottomrule
	
	\end{tabular}}
	\begin{tablenotes}
		\tiny
		\item $^1$Implemented at: https://github.com/clslabMSU $^2$http://sourceforge.net/projects/unibic/
		\item $^3$http://bicpams.com
	\end{tablenotes}
\end{table}



